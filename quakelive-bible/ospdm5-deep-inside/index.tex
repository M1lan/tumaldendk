% Created 2010-10-02 Sat 03:54
\documentclass[11pt]{article}
\usepackage[utf8]{inputenc}
\usepackage[T1]{fontenc}
\usepackage{fixltx2e}
\usepackage{graphicx}
\usepackage{longtable}
\usepackage{float}
\usepackage{wrapfig}
\usepackage{soul}
\usepackage{t1enc}
\usepackage{textcomp}
\usepackage{marvosym}
\usepackage{wasysym}
\usepackage{latexsym}
\usepackage{amssymb}
\usepackage{hyperref}
\tolerance=1000
\providecommand{\alert}[1]{\textbf{#1}}

\title{Deep Inside - OSPDM5}
\author{}
\date{}

\begin{document}

\maketitle

\setcounter{tocdepth}{3}
\tableofcontents
\vspace*{1cm}
 
Very legible map, similarly as by the others, timing and coordination  are in
focus in the first place.

There is only one power-up in it so every unused may lead a team which ignores
it to its doom.

It can be played in different ways, I'll try to show some tips and tricks :-) 

\section{Items Available}
\label{sec-1}
\subsection{Weapons}
\label{sec-1_1}

\begin{itemize}
\item 2 Shotgun (SG)
\item 1 Railgun (RG)
\item 1 Grenade Launcher (GL)
\item 1 Lightning Gun (LG)
\item 2 Rocket Launcher (RL)
\item 1 Plasma Gun (PG)
\end{itemize}
\subsection{Armors}
\label{sec-1_2}

\begin{itemize}
\item 1 Red Armor (RA)
\item 2 Yellow Armor (YA)
\item 7 Shards
\item 1 Mega Health (MH)
\item 1 50Health
\item 13 25 Health
\end{itemize}
\subsection{Powerups}
\label{sec-1_3}

\begin{itemize}
\item Quad
\end{itemize}


Quad - Railgun
 
RG

It is good to check your rail-gun and mega-health (The person who checks it
should only collect shards and mega-health), the enemies are vulnerable easily
and they cannot endanger you at a distance. I advise you to go for the mega-
health few seconds in advance and look around if there is no enemy hidden
there. In front of Quad-damage it is recommended when two players secure the
rail-gun as it is shown on the picture below. We pay attention to
sound especially which comes from jump-pad  - and we secure
the entrance into which the enemies from RA may jump to.

Quad
 
It is ideal when the group of players in front of Quad would be in positions as
it is shown on the picture below, but it isn`t always possibleJ.
Positions

No 1 is a RG boy he looks after the RG for his colleagues so they can have
it and shoot at distance to secure the rail-gun and mega-health. Position
 
No 2 is quite difficult, the difficulty depends upon how you play it. His task
is to always have in grasp when there will be RA and MH; eventually YA,  in
front of Quad he secures No1. The given person secures the jump-pad and
entrance from YA at position
 
No 3; apart from it, he checks the situation in central room and makes his
timings for 2 YA or mega. Position
 
No 4 gives a hand to number two and three it depends on where is the bulk
of enemies.

 
WTF? QUAD LOST ! CENTRAL \& RA ROOM

It is usual in ospdm5 that the position at rail-gun is often exchanged
therefore it is important to keep your head cool and start from
nothing what is RA in this case. Pay attention mainly to LG and RA; eventually, you try
to get some time MH. You can attack the quad directly with RA  - but
carefully as normally you have nothing available to shower the enemy at distance!
 


The picture above depicts the attack on quad when you don`t have the upper hand
at Quad -Rail-gun. Position No 1 takes care of that the enemy doesn`t  eat RA
or he eventually makes timing for LG (it is good to know the timing of mega as
well). He attacks straight through RA or carefully through jump-pad.  Position
No 2 secures the central room and the situation at quad. No 4 prevailingly
attacks through YA or directly from central room. Position No 3 is quite
complicated as you look after what is going on in the lower part of the
map. Usually he tries to get mega under control; it is good when No 1 gives a
hand at it. He tries to get on quad through the jump-pad or RG. The last
position No 4 must have in grasp over
those two YA, he attacks mainly through the upper YA
(it is usually effective in combination with a rocket-jumpJ) along with No 2.

\end{document}